\chapter{Related Works}

In recent years, the application of Convolutional Neural Networks (CNNs) has revolutionized the field of image analysis, 
offering promising solutions to various problems, including the detection and diagnosis of plant diseases. 
Despite the success of CNNs, the challenges inherent in precise plant disease diagnosis persist and continue to drive 
research in this domain.

The recognition and classification of plant diseases from images is a complex task due to variations in plant species, 
environmental conditions, and the diverse manifestations of diseases. Researchers have made significant strides in 
addressing these challenges, and several noteworthy contributions in this area have advanced our understanding and 
capabilities.

\textbf{F. Marzougui, M. Elleuch and M. Kherallah} in their paper titled 
\textit{A Deep CNN Approach for Plant Disease Detection} \cite{Marzougui} presented an ensemble solution for plant 
disease diagnosis. They leveraged variants of classifiers designed with pretrained neural network architectures. 
Their approach achieved remarkable results, including a weighted F1-score of 97.2\% on the test dataset. 
% LINK: https://ieeexplore.ieee.org/document/9300072

\textbf{Nishant Shelar, Suraj Shinde, Shubham Sawant, Shreyash Dhumal and Kausar Fakir} in their publication 
\textit{Plant Disease Detection Using Cnn} \cite{NishantShelar} Their paper aims to develop a Disease Recognition 
Model using leaf image classification and convolutional neural networks (CNNs) for precise plant disease detection. 
CNNs are specialized for image processing and recognition, making them suitable for this task. 
Their model demonstrated a weighted F1-score of 95.6\% on the final test set.
% LINK https://www.itm-conferences.org/articles/itmconf/pdf/2022/04/itmconf_icacc2022_03049.pdf

\textbf{Shi, T., Liu, Y., Zheng,} explored 
\textit{Recent advances in plant disease severity assessment using convolutional neural networks} \cite{Shi}.
This study reviews 16 CNN-based approaches for severity assessment, discusses dataset acquisition, evaluation metrics, 
and addresses challenges, offering research ideas and solutions for practical applications. Additionally, 
they incorporated a basic attention mechanism to enhance their model's performance. Their ensemble of neural 
networks achieved a weighted F1-score of 98.7\% on the final test dataset.
% LINK: https://www.nature.com/articles/s41598-023-29230-7#citeas

\textbf{Hassan, S.M.; Maji, A.K.; Jasi `nski, M.; Leonowicz, Z.; Jasi `nska} introduced 
\textit{Identification of Plant-Leaf Diseases Using CNN and Transfer-Learning Approach} \cite{Hassan}, 
they optimized the models by using depth-separable convolution, reducing parameters and computation. 
With training on a diverse dataset, the models achieved high accuracy rates, surpassing traditional methods, 
and outperformed other deep-learning models, showing promise for efficient real-time disease detection in agriculture.
% LINK https://www.sciencedirect.com/science/article/abs/pii/S0168169917311742

Our project, \textit{Precision in Plant Disease Diagnosis: A CNN Approach to Enhance Agricultural Practices}, 
contributes to this body of work by evaluating the effectiveness of CNN models in plant disease detection. 
We aim to determine which approach, whether building a CNN from scratch or utilizing pre-trained weights, 
offers superior accuracy in plant disease diagnosis. Our efforts are geared towards advancing agricultural 
practices and promoting healthier crop yields.
